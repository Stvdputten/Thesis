\chapter{Research Background}  \label{ch:theory}

TODO Finish this chapter \ldots\ and the rest!


\section{definitions}

\section{Related work}  

This section contains the research background relevant to our current research. Research conducted in 
Latent Dirichlet Allocation (LDA) was introduced in 2003 by \cite{Blei2003LatentAllocation}. LDA is used for topic modeling on a corpera of documents. A lot of research has been performed in this area for different uses. The application of LDA ranges from image recognition to simple document topic discovery. LDA  In the following sections these different applications will be discussed.






\section{Buiding blocks}
To even analyse and compute the system logs discussed in this research existing frameworks and libraries had to be used. Software is required to perform extracting, loading and transforming the data in a useable form.

\subsection{Software}
In this section we discuss the various tools used. There a lot more tools, but these tools are chosen mainly because they are open source and relative easy to use.

\begin{enumerate}
    \item \textbf{Hadoop} $($\url{http://hadoop.apache.org/}$)$ \\
    The Apache Hadoop software library is a framework that allows for the distributed processing of large data sets across clusters of computers using simple programming models. 
    \begin{enumerate}
        \item \textbf{Apache Spark} $($\url{https://spark.apache.org/}$)$\\
        Apache Spark is a fast and general engine for large-scale data processing. Spark lets users compute in multiple languages.
        \item \textbf{HDFS} $($\url{http://hadoop.apache.org/}$)$ \\
         A distributed file system that provides high-throughput access to application data.
         \item \textbf{Apache Zeppelin} $($\url{https://zeppelin.apache.org/}$)$ \\
         Web-based notebook that enables data-driven, interactive data analytics and collaborative documents with SQL, Scala and more.
    \end{enumerate}
    
    \item \textbf{Scikit-learn} $($\url{http://scikit-learn.org/}$)$ \\
    The Scikit-learn package contains tools for effici\"ent data mining and data analysis with machine learning in Python.
    \begin{enumerate}
        \item \textbf{Pandas} $($\url{http://pandas.pydata.org/}$)$ \\
        Pandas is a library providing high-performance, easy-to-use data structures and data analysis tools for the Python programming language.
        \item \textbf{Numpy} $($\url{http://www.numpy.org/}$)$ \\
        Numpy is a scientific package with Python for powerful array objects, functions and lots of mathematical capabilities.
        \item \textbf{Matplotlib} $($\url{http://matplotlib.org/}$)$ \\
        Matplotlib is a 2D Python plotting library very similar to MATLAB.
        
    \end{enumerate}
    
\end{enumerate}

\section{Dataset}
The dataset was provided by Capgemini containing vrops, metering and syslogs. Eventually syslogs were chose as suitable data set for analysis. 

\section{Data mining}


\section{Machine learning}

