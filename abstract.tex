%\addcontentsline{toc}{chapter}{Abstract}
\newpage
\begin{abstract}
Our data is a collection of server logs. The servers logs have been aggregated but have no labels to indicate what type the servers logs are. This thesis focuses on the discovery and clustering of these server logs. The focus has been on logs with the term 'error'. 
The research makes use of the unsupervised machine learning technique Latent Dirichlet Allocation (LDA). We extract the server logs, transform them using the standard data preprocessing pipeline. With that we created a dataset which we can call the corpus and the logs are the documents. 
Using the best practices from Blei the founder of LDA, we create multiple models only varying in the topic count. The models are evaluated using multiple metrics. Topic modelling can distinguish itself by being one of the few machine learning techniques which depends on human readability of its models. We take a look at the topics generated by the models and conclude that a human has a hard time understanding the topics. Clustering the documents based on their highest probable topic, shows that models only have a few dominant topics where the bulk of the documents go. The clustering has a great performance based on silhouette coefficient on lower levels. At the end of the thesis we do not recommend topic modelling for latent topic discovery on server logs. Topic modelling is not human readable on server logs and applying semantic analysis metrics does not help a lot. The clustering however appears to create solid clusters when using low topic counts.
\end{abstract}

