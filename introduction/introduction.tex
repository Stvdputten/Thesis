\chapter{Introductie}  \label{ch:introduction}
In this chapter we give an introduction to the problem addressed in this thesis.

 
 
Introduction 
 
The intention of this research started with analyzing the Big data from system logs to help create a system for predicting hardware and software failure for maintenance and automatic self-healing. 
The current research of Big data makes this a suitable problem to solve through recent machine learning techniques. During the time spent on this research challenges were met and identified for realizing this goal and ended with the usage of Natural Language Processing (NLP) and unsupervised learning.  The untapped amount of raw data makes it possible for many more application, but in further paragraphs it will be made clear why NLP was chosen and what more could be applied on this Big data. 
 
 
 
 
 
 
 
 
 
 
 
\section{Purpose}



\section{Applications}
Chapters may include sections.

To make sure that this document renders correctly, execute these commands:
\begin{quote}
\begin{verbatim}
pdflatex thesis
bibtex thesis
pdflatex thesis
pdflatex thesis
\end{verbatim}
\end{quote}
Here, the \verb|pdflatex| command may need to be executed three times in order to generate the table of contents and so on. 
Note that a good thesis has figures and tables; examples can be found in Figure~\ref{fig:afigure} and Table~\ref{tab:atable}. And every thesis has references, like~\cite{brilliantgift15}.

\begin{figure}
\begin{center}
\input{figures/afigure}
\end{center}
\caption{Every thesis should have figures.\label{fig:afigure}}
\end{figure}

\begin{table}
\begin{center}
\begin{tabular}{ll}
Column A & Column B\\
\hline
Point 1 & Good\\
Point 2 & Bad\\
\end{tabular}
\end{center}
\caption{Every thesis should have tables.\label{tab:atable}}
\end{table}

Final reminder: this template is just an example, if you want you can make adjustments; also discuss with your supervisor which layout he or she likes. But the front page should be as it is now.

TODO: quite a lot!

\section{Thesis Overview}
It is recommended to end the introduction with an overview of the thesis. This chapter contains the introduction; Chapter~\ref{ch:definitions} includes the definitions; Chapter~\ref{ch:relatedwork} discusses related work; Chapter~\ref{ch:evaluation} evaluates the contributions; Chapter~\ref{ch:conclusions} concludes.

Also make a nice sentence with ``bachelor thesis'', LIACS and the names of the supervisors.

