\chapter{Introduction} \label{ch:introduction}


\section{General introduction}\label{introduction:Generalinformation}
This thesis is made with the collaboration of Capgemini. Capgemini is an international IT consultancy firm, offering customers multiple services. One of these services is the big data lake, which allows enormous amounts of data to be stored for further use. This data lake , build with the Hadoop framework, contains event logs. Capgemini allowed us access to their Big data lake, containing millions of server logs of their customers and systems. 

With enough creativity and time we could use such a source of data to infinite use cases. The research started with the analyses of the data but eventually led to topic modelling, which can be classified as a form of unsupervised machine learning. Topic modelling can be described as the extraction of latent patterns, a.k.a. hidden topics, from data through semantic analysis. Readers who are interested  why topic modelling has been chosen are encouraged to read section \ref{conclusion:discussion}. 

--todo--

\begin{comment}
The intention of this research started with analysing e system logs to help create a model for predicting hardware and software failure for maintenance and automatic self-healing. The huge amount of system logs available from a variety of systems brought the question how to analyse and make use of the logs to predict hardware and software failure.

Current research of big data makes this a suitable problem to solve through recent machine learning techniques. 
During the time spent on this research challenges were met and identified for realising this goal and ended with the usage of Natural Language Processing (NLP) and unsupervised learning.  The untapped amount of raw data makes it possible for many more application, but in further paragraphs it will be made clear why NLP was chosen and what more could be applied on this Big data problem.

 \end{comment}
 
\section{Motivation/probleemstelling}

Topic modelling is a hot topic in data science. 
System logs are a primary source for the detection of problems in large computer systems, like data warehouses. While domain experts can be used to detect and fix the problems detected, this can be difficult and time consuming. Machine learning techniques like topic modelling make it possible to develop models to extract these latent patterns from these system logs. While topic modelling is normally used in for large text corpera, recent research in the field of short text clustering and twitter tweets clustering are similar enough to by applicable for system logs. An interesting application which has not yet been touched a lot through unsupervised machine learning techniques.

\section{Objectives}


\section{Research question}
Can we use topic modelling to classify and cluster error messages?

What are the optimal parameters?

subquestions:
Why is LDA suitable for this type of data 

How do the parameters influence the models performance?

What are the pro's and con's of using LDA?

What other methods are available to solve this error clustering?



\begin{comment}

How do we define similair messages?

Can we cluster these error messages to find patterns?

Wat ik zou verwachten in je scriptie zijn de volgende topics
•	Wat is het probleem (de probleemstelling)?
•	Welke mogelijkheden zijn er om dit probleem op te lossen?
•	Welke methode heb je gekozen, en vooral uitleggen waarom deze methode volgens jou de beste is?
•	Hoe heb je vastgesteld dat de gekozen oplossing de beste is?
•	Wat is de uitkomst?
•	Wat zijn de voor en nadelen van het gekozen model, wat zijn de beperkingen, wat is de optimale modelering en waarom?
Zijn deze onderwerpen voldoende afgedicht in onderstaande structuur?

Ik ben veel meer geïnteresseerd in de onderbouwing:
•	Wat is de (onze) probleem omschrijving
Finding structures in syslogs to cluster undiscovered syslogs with errors
•	Waarom kies je LDA om dit probleem te lijf te gaan
o	Pro’s / con’s
o	Alternatieven
•	Hoe moet LDA gebruikt worden
o	Welke specifieke tuning heb je gebruikt
o	Hoe beinvloed de aanpassingen van parameters de uitkomst


\textbf{Can we make a reliable model for error detection in system logs?}

 \end{comment}
 
\section{Thesis Overview}

\begin{comment}
It is recommended to end the introduction with an overview of the thesis. This chapter contains the introduction; Chapter~\ref{ch:definitions} includes the definitions; Chapter~\ref{ch:relatedwork} discusses related work; Chapter~\ref{ch:evaluation} evaluates the contributions; Chapter~\ref{ch:conclusions} concludes.

Also make a nice sentence with ``bachelor thesis'', LIACS and the names of the supervisors.

\end{comment}