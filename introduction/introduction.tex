\chapter{Introduction} \label{ch:introduction}
The intention of this research started with analysing the system logs to help create a model for predicting hardware and software failure for maintenance and automatic self-healing. The huge amount of system logs available from a variety of systems brought the question how to analyse and make use of the logs to predict hardware and software failure.

Current research of big data makes this a suitable problem to solve through recent machine learning techniques. 
During the time spent on this research challenges were met and identified for realising this goal and ended with the usage of Natural Language Processing (NLP) and unsupervised learning.  The untapped amount of raw data makes it possible for many more application, but in further paragraphs it will be made clear why NLP was chosen and what more could be applied on this Big data problem.
 
 
\section{Motivation}


\section{Objectives}


\section{Research question}
How do we define similair messages?

Can we cluster these error messages to find patterns?


\textbf{Can we make a reliable model for error detection in system logs?}

\section{Eva}
 
\section{Thesis Overview}
It is recommended to end the introduction with an overview of the thesis. This chapter contains the introduction; Chapter~\ref{ch:definitions} includes the definitions; Chapter~\ref{ch:relatedwork} discusses related work; Chapter~\ref{ch:evaluation} evaluates the contributions; Chapter~\ref{ch:conclusions} concludes.

Also make a nice sentence with ``bachelor thesis'', LIACS and the names of the supervisors.

