\chapter{Conclusions} \label{ch:conclusions}

\section {Future work}
This research took only a small possible path in the finite amount of paths. 
\subsection{Scalability}
\subsection{Patterns}
\subsection{Predictive analysis}

\section{Discussion}
Although the discussed research has been mainly focused around topic modelling (LDA), the original research question started with a related but different subject. During this chapter i\'d like to discuss the original focus of predictive maintenance, as a lot of time has also been put in researching this difficult and challenging task. The points that will be discussed are the general application of predictive maintenance and process, the primary tools used, the identified challenges particular to our data set and environment and a recommendation for future work on this subject. This part has probably taken 40\% of the total time spent to complete this Thesis.

\subsection{Predictive maintenance}
With the combined application of machine learning and big data companies try to anticipate when machine hardware failure are due to occur. Predicting instead of reacting to problems saves time and money and allows for a better customer experience. It\'s not hard to imagine why companies like Intel \cite{AjayChandramoulyRavindraNarkhedeVijayMungaraGuillermoRueda2013ReducingAnalytics} or Google have already been researching the possibility of big data for this problem. Intel gathered event logs from 
The original data set discussed in chapter \ref{ch:introduction} has a lot in common with earlier research done on Predictive maintenance \cite{Sipos2014Log-basedMaintenance}

\subsection{The process}

\subsection{Primary tools}

\subsection{Challenges}

\subsection{Recommendation}
The earlier discussed challenges of predictive maintenance and the dependency on domain experts makes researching predictive maintenance time consuming. 